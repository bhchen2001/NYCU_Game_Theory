%%%%%%%%%%%%%%%%%%%%%%%%%%%%%%%%%%%%%%%%%
% Stylish Curriculum Vitae
% LaTeX Template
% Version 1.1 (September 10, 2021)
%
% This template originates from:
% https://www.LaTeXTemplates.com
%
% Authors:
% Stefano (https://www.kindoblue.nl)
% Vel (vel@LaTeXTemplates.com)
%
% License:
% CC BY-NC-SA 4.0 (https://creativecommons.org/licenses/by-nc-sa/4.0/)
%
%%%%%%%%%%%%%%%%%%%%%%%%%%%%%%%%%%%%%%%%%
% !TEX program = xelatex
\documentclass[a4paper, oneside, final, 12pt]{scrartcl} % Paper options using the scrartcl class

\usepackage{fontspec} % for other font
\usepackage{xeCJK} % for chinese font
\usepackage{hyperref} % for hyper web link
\usepackage{multirow} % for tabular table in learning progress
\usepackage{graphicx} % for image insersion
\usepackage[export]{adjustbox} % for image frame
\usepackage{setspace}
\usepackage{array}
% Define typographic struts, as suggested by Claudio Beccari
%   in an article in TeX and TUG News, Vol. 2, 1993.
\usepackage{mathptmx}
\usepackage{scrlayer-scrpage} % Provides headers and footers configuration
\usepackage{titlesec} % Allows creating custom \section's
\usepackage{marvosym} % Allows the use of symbols
\usepackage{tabularx,colortbl} % Advanced table configurations
% \usepackage{ebgaramond} % Use the EB Garamond font
\usepackage{microtype} % To enable letterspacing
\usepackage{pdfpages} % for showing pdf
\usepackage{pdflscape}
\usepackage{enumitem}
\usepackage{subcaption}
\usepackage{listings}   % highlight the python code
\usepackage{xcolor}
% setup the margin
\usepackage[top=3cm, bottom=3cm]{geometry}

% set the style of listing code
\definecolor{codegreen}{rgb}{0,0.6,0}
\definecolor{codegray}{rgb}{0.5,0.5,0.5}
\definecolor{codepurple}{rgb}{0.58,0,0.82}
\definecolor{backcolour}{rgb}{0.95,0.95,0.92}

\lstdefinestyle{mystyle}{
    backgroundcolor=\color{backcolour},   
    commentstyle=\color{codegreen},
    keywordstyle=\color{magenta},
    numberstyle=\tiny\color{codegray},
    stringstyle=\color{codepurple},
    basicstyle=\ttfamily\footnotesize,
    breakatwhitespace=true,         
    breaklines=true,                 
    captionpos=b,                    
    keepspaces=true,                 
    numbers=left,                    
    numbersep=5pt,                  
    showspaces=false,                
    showstringspaces=false,
    showtabs=false,                  
    tabsize=2
}

\lstset{style=mystyle}

% set chinese and english font
\setmainfont{Times New Roman}
\setCJKmainfont[AutoFakeBold=true, AutoFakeSlant=true]{標楷體}

\titleformat{\section}{\Large\raggedright\bfseries}{}{0em}{}[\titlerule] % Section formatting
\titleformat{\subsection}{\large\raggedright\bfseries}{}{0em}{}

% \pagestyle{scrheadings} % Print the headers and footers on all pages

\addtolength{\voffset}{-0.8in} % Adjust the vertical offset - less whitespace at the top of the page
\addtolength{\textheight}{10cm} % Adjust the text height - less whitespace at the bottom of the page

% enable bold and slant chinese font
% \xeCJKsetup{AutoFakeBold=true, AutoFakeSlant=true}

% set the space at the front of paragraph
\setlength{\parindent}{2em}

% disable page number
\pagenumbering{gobble}

\newcommand{\gray}{\rowcolor[gray]{.90}} % Custom highlighting for the work experience and education sections
\newcommand{\Tstrut}{\rule{0pt}{2.6ex}}         % = `top' strut
\newcommand{\Bstrut}{\rule[-0.9ex]{0pt}{0pt}}   % = `bottom' strut
\newcommand{\Tstruth}{\rule{0pt}{4ex}}         % = `top' strut for header
\newcommand{\Bstruth}{\rule[-2.5ex]{0pt}{0pt}}   % = `bottom' strut for header

% marco
% \def\projectname {應用神經網路架構搜尋技術提升Youbike單車系統之車流預測準確率}

%----------------------------------------------------------------------------------------
%	FOOTER SECTION
%----------------------------------------------------------------------------------------

% \renewcommand{\headfont}{\normalfont\rmfamily\scshape} % Font settings for footer

% \cofoot{
% \fontsize{12.5}{17}\selectfont % Letter spacing and font size

% \textls[150]{123 Broadway {\large\textperiodcentered} City {\large\textperiodcentered} Country 12345}\\ % Your mailing address
% {\Large\Letter} \textls[150]{john@smith.com \ {\Large\Telefon} (000) 111-1111} % Your email address and phone number
% }

%----------------------------------------------------------------------------------------
\begin{document}

%----------------------------------------------------------------------------------------
%	HEADER SECTION
%----------------------------------------------------------------------------------------


\begin{center}
    {\fontsize{18}{30}\textbf{Game Theory Assignment 1 \\ Fictitious Play}}
\end{center}

\begin{center}
  Bo-Han Chen (陳柏翰) \\
  Student ID:312551074 \\
  bhchen312551074.cs12@nycu.edu.tw
\end{center}

\section{Experiment Environment}

The experiment environment is based on
\begin{itemize}
  \item Windows 10 Education 22H2
  \item Python 3.12 with package
  \begin{itemize}
    \item Numpy 1.26.0
  \end{itemize}
\end{itemize}

\section{Source Code Overview}

\subsection{Setup}

\begingroup
\raggedright
The \emph{FictitiousPlay} object will be constructed with \emph{gameType} 
(which game does the user want to simulate by fictitious play model),
\emph{iterTime} (the number of game round), \emph{numPrev} 
(the sum of prior belief, means how much rounds does two player already played before round 0),
and \emph{actionTh} (threshold for stopping the iteration if two players keep choosing the same best response).

\begin{lstlisting}[language=Python]
  class FictitiousPlay():
    def __init__(self, gameType, iterTime, numStrategy = 2, numPrev = 1000):
      self.gameType = gameType
      self.iterTime = iterTime
      # fictitious play is designed for games with 2 player
      self.numPlayer = 2
      # the threshold of repeated action
      # if two player repeat the same action for more than actionTh times, stop the iteration
      self.actionTh = 3000
      self.numStrategy = numStrategy
      self.numPrev = numPrev
      self.payoffMat = np.zeros([self.numPlayer * self.numStrategy, self.numStrategy])
      self.beliefMat = np.zeros([self.numPlayer, self.numStrategy])
      self.actionSet = np.zeros([self.numPlayer, iterTime + 1], dtype = np.uint8)
      self.payoffSet = np.zeros([self.numPlayer, iterTime, self.numStrategy])
\end{lstlisting}

After knowing the game type, \emph{game\_selection} function will setup the payoff matrix.

\begin{lstlisting}[language=Python]
  def game_selection(self):
    # set the payoff matrix of each player
    if self.gameType == "Q1":
        self.payoffMat[:self.numStrategy, :] = [[-1, 1], [0, 3]]
        self.payoffMat[self.numStrategy:, :] = [[-1, 1], [0, 3]]
\end{lstlisting}

\newpage

The initial belief can be defined by programmer or assigned randomly.

\begin{lstlisting}[language=Python]
  def init_belief(self):
    # setup the prior belief of each player
    if self.numPrev == -1:
      self.beliefMat = np.array([[500, 500], [500, 500]])
    else:
      playTime = random.uniform(0, self.numPrev)
      self.beliefMat[0, :] = [playTime, (self.numPrev - playTime)]
      playTime = random.uniform(0, self.numPrev)
      self.beliefMat[1, :] = [playTime, (self.numPrev - playTime)]
\end{lstlisting}

The \emph{payoff\_cal} function will calculate the payoff by player's belief and payoff.

\begin{lstlisting}[language=Python]
  def payoff_cal(self, belief, payoffMat):
    # calculate the payoff of player by belief
    payoff = np.zeros([self.numStrategy])
    # beliefNor = belief / np.sum(belief)
    beliefNor = belief
    for idx in range(self.numStrategy):
        payoff[idx] = np.sum(np.multiply(beliefNor, payoffMat[idx]))
    return payoff
\end{lstlisting}

For each iteration in \emph{play\_loop} function, the payoff of each player will be calculated to
find the best response (randomly choose if the payoffs are same for each strategy), 
and the belief will be updated according to the other player's strategy.
The iteration may be stopped due to the repeated selection of the same best response for two players.

\begin{lstlisting}[language=Python]
  def play_loop(self):
    pre_action1 = -1; pre_action2 = -1
    for iter in range(self.iterTime):
      #  calculate the payoff to find the best response
      self.payoffSet[0, iter, :] = self.payoff_cal(self.beliefMat[0, :], self.payoffMat[:2, :])
      self.payoffSet[1, iter, :] = self.payoff_cal(self.beliefMat[1, :], self.payoffMat[2:, :])

      self.logger(iter, self.actionSet[:, iter], self.beliefMat, self.payoffSet[:, iter, :])

      # same payoff for each strategy --> choose randomly
      if self.payoffSet[0, iter, 0] == self.payoffSet[0, iter, 1]:
        action1 = randint(0, 1)
      # different payoff--> choose best response
      else:
        action1 = int(np.argmax(self.payoffSet[0, iter, :]))
      self.actionSet[0, iter + 1] = action1
      if self.payoffSet[1, iter, 0] == self.payoffSet[1, iter, 1]:
        action2 = randint(0, 1)
      else:
        action2 = int(np.argmax(self.payoffSet[1, iter, :]))
      self.actionSet[1, iter + 1] = action2

      # update the other player's belief
      self.beliefMat[1, self.actionSet[0, iter + 1]] += 1
      self.beliefMat[0, self.actionSet[1, iter + 1]] += 1

      if action1 == pre_action1 and action2 == pre_action2:
        self.actionTh -= 1
        if self.actionTh == 0:
          # print("early stop with repeated best response")
          self.logger(iter, self.actionSet[:, iter], self.beliefMat,
                      self.payoffSet[:, iter, :], last = True)
          break
      pre_action1, pre_action2 = action1, action2
\end{lstlisting}

\endgroup

\newpage

\section{Questions}

\subsection{Q1. One pure-strategy Nash Equilibrium}

The following block shows the result of Q1 by fictitious play, 
for each iteration the program will print out the action, belief and payoff of each player.
When the game end (or converge), the strategy distribution will be shown,
which can be use to find the mixed or pure strategy of the game
(the strategy distribution only considers the actions after round 0, 
which means the prior belief is ignored).

\begin{lstlisting}
                       Round 0
  belief of player1: [264.93 735.07], player2: [231.05 768.95]
  payoff of player1: [ 470.15 2205.22], player2: [ 537.9  2306.86]
  ==================================================================
                       Round 1
  action of player1: 1, palyer2: 1
  belief of player1: [264.93 736.07], player2: [231.05 769.95]
  payoff of player1: [ 471.15 2208.22], player2: [ 538.9  2309.86]
  ==================================================================
  .
  .
  ==================================================================
                       Round 3000
  action of player1: 1, palyer2: 1
  belief of player1: [ 264.93 3736.07], player2: [ 231.05 3769.95]
  payoff of player1: [ 3470.15 11205.22], player2: [ 3537.9  11306.86]
  ==================================================================
  strategy distribution of player1: [0. 1.], player2: [0. 1.]
\end{lstlisting}

\begingroup
\raggedright
The pure-strategy NE $(r_2, c_2)$ can be found by fictitious paly.
For player1, the best response will always be $r_2$ 
not matter what strategy player2 chooses, and vice versa.
\endgroup

\subsection{Q2. Two or more pure-strategy NE}

\begin{lstlisting}
                       Round 0
  belief of player1: [637.08 362.92], player2: [551.77 448.23]
  payoff of player1: [1637.08 1088.77], player2: [1551.77 1344.7 ]
  ==================================================================
                       Round 1
  action of player1: 0, palyer2: 0
  belief of player1: [638.08 362.92], player2: [552.77 448.23]
  payoff of player1: [1639.08 1088.77], player2: [1553.77 1344.7 ]
  ==================================================================
  .
  .
  ==================================================================
                       Round 3000
  action of player1: 0, palyer2: 0
  belief of player1: [3638.08  362.92], player2: [3552.77  448.23]
  payoff of player1: [7637.08 1088.77], player2: [7551.77 1344.7 ]
  ==================================================================
  strategy distribution of player1: [1. 0.], player2: [1. 0.]
\end{lstlisting}

\newpage

\begin{lstlisting}
                       Round 0
  belief of player1: [297.91 702.09], player2: [487.34 512.66]
  payoff of player1: [1297.91 2106.28], player2: [1487.34 1537.97]
  ==================================================================
                       Round 1
  action of player1: 1, palyer2: 1
  belief of player1: [297.91 703.09], player2: [487.34 513.66]
  payoff of player1: [1298.91 2109.28], player2: [1488.34 1540.97]
  ==================================================================
  .
  .
  ==================================================================
                       Round 3000
  action of player1: 1, palyer2: 1
  belief of player1: [ 297.91 3703.09], player2: [ 487.34 3513.66]
  payoff of player1: [ 4297.91 11106.28], player2: [ 4487.34 10537.97]
  ==================================================================
  strategy distribution of player1: [0. 1.], player2: [0. 1.]
\end{lstlisting}

\begingroup
\raggedright
The pure-strategy NE $(r_1, c_1)$ and $(r_2, c_2)$ can be found by fictitious paly.
No matter what the initial belief is, 
at some point the game will lead the belief of both players to $(x, y), x=y$,
and the result will have two situation:
\begin{enumerate}
  \item Once the belief of both player are $(x, y), x > y$, 
  player1 will keep choosing $r_1$ rather than $r_2$, 
  and player2 will keep choosing $c_1$ rather than $c_2$, 
  which result in pure-strategy NE $(r_1, c_1)$.
  \item Once the belief of both player are $(x, y), x < y$, 
  player1 will keep choosing $r_2$ rather than $r_1$, 
  and player2 will keep choosing $c_2$ rather than $c_1$, 
  which result in pure-strategy NE $(r_2, c_2)$.
\end{enumerate}

\endgroup

\subsection{Q3: Two or more pure-strategy NE (Conti.)}

\begin{lstlisting}
                        Round 0
  belief of player1: [179.43 820.57], player2: [741.07 258.93]
  payoff of player1: [179.43   0.  ], player2: [741.07   0.  ]
  ==================================================================
                        Round 1
  action of player1: 0, palyer2: 0
  belief of player1: [180.43 820.57], player2: [742.07 258.93]
  payoff of player1: [180.43   0.  ], player2: [742.07   0.  ]
  ==================================================================
  .
  .
  ==================================================================
                        Round 3000
  action of player1: 0, palyer2: 0
  belief of player1: [3180.43  820.57], player2: [3742.07  258.93]
  payoff of player1: [3179.43    0.  ], player2: [3741.07    0.  ]
  ==================================================================
  strategy distribution of player1: [1. 0.], player2: [1. 0.]
\end{lstlisting}

\begingroup
\raggedright
Only one pure-strategy NE $(r_1, c_1)$ can be found by fictitious paly.
Defining $(b_{c1}, b_{c2}), b_{c1} > 0,  b_{c2} > 0$ as player1's initial belief,
$(b_{r1}, b_{r2}), b_{r1} > 0,  b_{r2} > 0$ as player2's initial belief,
the payoff of $r_1$ ($c_1$) will always be larger than $r_2$ ($c_2$),
so the result will only converge to NE $(r_1, c_1)$.
\endgroup

\newpage

\subsection{Q4: Mixed-Strategy Nash Equilibrium}

\begin{lstlisting}
                        Round 0
  belief of player1: [336.21 663.79], player2: [664.74 335.26]
  payoff of player1: [1327.58  672.42], player2: [ 664.74 1341.03]
  ==================================================================
                        Round 1
  action of player1: 0, palyer2: 1
  belief of player1: [336.21 664.79], player2: [665.74 335.26]
  payoff of player1: [1329.58  672.42], player2: [ 665.74 1341.03]
  ==================================================================
  .
  .
  ==================================================================
                        Round 4999
  action of player1: 0, palyer2: 1
  belief of player1: [2864.21 3134.79], player2: [4647.74 1351.26]
  payoff of player1: [6269.58 5728.42], player2: [4647.74 5405.03]
  ==================================================================
  strategy distribution of player1: [0.8 0.2], player2: [0.51 0.49]
\end{lstlisting}

\begingroup
\raggedright
The mixed-strategy NE $P(r_1) = \frac{4}{5}, P(r_2) = \frac{1}{5}, 
P(c_1) = \frac{1}{2}, P(c_2) = \frac{1}{2}$ can be found by fictitious paly.
The best response of two player can be analysis by four situation:
\begin{enumerate}
  \item belief $b_{c1} > b_{c2}, b_{r1} > b_{r2}$, 
  \textbf{player1 will choose $r_2$ as best response}, player2 will choose $c_1$, 
  which lead to situation 2
  \item belief $b_{c1} > b_{c2}, b_{r1} < b_{r2}$,
  player1 will choose $r_2$ as best response, \textbf{player2 will choose $c_2$}, 
  which lead to situation 3
  \item belief $b_{c1} < b_{c2}, b_{r1} < b_{r2}$,
  \textbf{player1 will choose $r_1$ as best response}, player2 will choose $c_2$, 
  which lead to situation 4
  \item belief $b_{c1} < b_{c2}, b_{r1} > b_{r2}$,
  player1 will choose $r_1$ as best response, \textbf{player2 will choose $c_1$}, 
  which lead to back to situation 1
\end{enumerate}
With large number of iteration and the best response cycle mentioned above, 
the mixed-strategy NE can be found by fictitious play.
\endgroup

\subsection{Q5: Best-reply path}

\begin{lstlisting}
                        Round 0
  belief of player1: [846.93 153.07], player2: [768.67 231.33]
  payoff of player1: [153.07 846.93], player2: [768.67 231.33]
  ==================================================================
                        Round 1
  action of player1: 1, palyer2: 0
  belief of player1: [847.93 153.07], player2: [768.67 232.33]
  payoff of player1: [153.07 847.93], player2: [768.67 232.33]
  ==================================================================
  .
  .
  ==================================================================
                        Round 4999
  action of player1: 1, palyer2: 0
  belief of player1: [3380.93 2618.07], player2: [3235.67 2763.33]
  payoff of player1: [2618.07 3380.93], player2: [3235.67 2763.33]
  ==================================================================
  strategy distribution of player1: [0.49 0.51], player2: [0.51 0.49]
\end{lstlisting}

\begingroup
\raggedright
The mixed-strategy NE $P(r_1) = \frac{4}{5}, P(r_2) = \frac{1}{5}, 
P(c_1) = \frac{1}{2}, P(c_2) = \frac{1}{2}$ can be found by fictitious paly.
The best response of two player can be analysis by four situation:

\newpage

\begin{enumerate}
  \item belief $b_{c1} > b_{c2}, b_{r1} > b_{r2}$, 
  \textbf{player1 will choose $r_2$ as best response}, player2 will choose $c_1$, 
  which lead to situation 2
  \item belief $b_{c1} > b_{c2}, b_{r1} < b_{r2}$,
  player1 will choose $r_2$ as best response, \textbf{player2 will choose $c_2$}, 
  which lead to situation 3
  \item belief $b_{c1} < b_{c2}, b_{r1} < b_{r2}$,
  \textbf{player1 will choose $r_1$ as best response}, player2 will choose $c_2$, 
  which lead to situation 4
  \item belief $b_{c1} < b_{c2}, b_{r1} > b_{r2}$,
  player1 will choose $r_1$ as best response, \textbf{player2 will choose $c_1$}, 
  which lead to back to situation 1
\end{enumerate}
With large number of iteration and the best response cycle mentioned above, 
the mixed-strategy NE can be found by fictitious play.
\endgroup

\subsection{Q6: Pure-Coordination Game}

\begin{lstlisting}
                        Round 0
  belief of player1: [859.82 140.18], player2: [814.87 185.13]
  payoff of player1: [8598.24 1401.76], player2: [8148.69 1851.31]
  ==================================================================
                        Round 1
  action of player1: 0, palyer2: 0
  belief of player1: [860.82 140.18], player2: [815.87 185.13]
  payoff of player1: [8608.24 1401.76], player2: [8158.69 1851.31]
  ==================================================================
  .
  .
  ==================================================================
                        Round 4999
  action of player1: 0, palyer2: 0
  belief of player1: [5858.82  140.18], player2: [5813.87  185.13]
  payoff of player1: [58588.24  1401.76], player2: [58138.69  1851.31]
  ==================================================================
  strategy distribution of player1: [1. 0.], player2: [1. 0.]
\end{lstlisting}

\begin{lstlisting}
                        Round 0
  belief of player1: [224.77 775.23], player2: [ 57.43 942.57]
  payoff of player1: [2247.67 7752.33], player2: [ 574.34 9425.66]
  ==================================================================
                        Round 1
  action of player1: 1, palyer2: 1
  belief of player1: [224.77 776.23], player2: [ 57.43 943.57]
  payoff of player1: [2247.67 7762.33], player2: [ 574.34 9435.66]
  ==================================================================
  .
  .
  ==================================================================
                        Round 4999
  action of player1: 1, palyer2: 1
  belief of player1: [ 224.77 5774.23], player2: [  57.43 5941.57]
  payoff of player1: [ 2247.67 57742.33], player2: [  574.34 59415.66]
  ==================================================================
  strategy distribution of player1: [0. 1.], player2: [0. 1.]
\end{lstlisting}

\begingroup
\raggedright
If the difference of initial payoff is larger than 10 
($|payoff_{r1} - payoff_{r2}| > 10$ and $|payoff_{c1} - payoff_{c2}| > 10$)
The pure-strategy NE $(r_1, c_1)$ and $(r_2, c_2)$ can be found by fictitious paly.
The result will have two situation:

\newpage

\begin{enumerate}
  \item Once the belief of both players are $(x, y), x > y$, 
  player1 will keep choosing $r_1$ rather than $r_2$, 
  and player2 will keep choosing $c_1$ rather than $c_2$, 
  which result in pure-strategy NE $(r_1, c_1)$.
  \item Once the belief of both players are $(x, y), x < y$, 
  player1 will keep choosing $r_2$ rather than $r_1$, 
  and player2 will keep choosing $c_2$ rather than $c_1$, 
  which result in pure-strategy NE $(r_2, c_2)$.
\end{enumerate}

\endgroup

\begin{lstlisting}
                        Round 0
  belief of player1: [500.01 499.99], player2: [499.96 500.04]
  payoff of player1: [5000.1 4999.9], player2: [4999.6 5000.4]
  ==================================================================
                        Round 1
  action of player1: 0, palyer2: 1
  belief of player1: [500.01 500.99], player2: [500.96 500.04]
  payoff of player1: [5000.1 5009.9], player2: [5009.6 5000.4]
  ==================================================================
  .
  .
  ==================================================================
                        Round 4999
  action of player1: 0, palyer2: 1
  belief of player1: [2999.01 2999.99], player2: [2999.96 2999.04]
  payoff of player1: [29990.1 29999.9], player2: [29999.6 29990.4]
  ==================================================================
  strategy distribution of player1: [0.5 0.5], player2: [0.5 0.5]
\end{lstlisting}

\begingroup
\raggedright
If the difference of initial payoff is lower than 10 
($|payoff_{r1} - payoff_{r2}| < 10$ and $|payoff_{c1} - payoff_{c2}| < 10$)
and initial strategy distribution of two player is 
$b_{r1} > b_{r2}, b_{c1} < b_{c2}$  or  $b_{r1} < b_{r2}, b_{c1} > b_{c2}$, 
the mixed-strategy NE $P(r_1) = \frac{1}{2}, P(r_2) = \frac{1}{2}, 
P(c_1) = \frac{1}{2}, P(c_2) = \frac{1}{2}$ can be found by fictitious paly.
Since the best response of each iteration will be lead to the cycle 
between $(r_1, c_2)$ and $(r_2, c_1)$.
\endgroup

\section{Q7: Anti-Coordination game}

\begin{lstlisting}
                        Round 0
  belief of player1: [179.25 820.75], player2: [559.01 440.99]
  payoff of player1: [820.75 179.25], player2: [440.99 559.01]
  ==================================================================
                        Round 1
  action of player1: 0, palyer2: 1
  belief of player1: [179.25 821.75], player2: [560.01 440.99]
  payoff of player1: [821.75 179.25], player2: [440.99 560.01]
  ==================================================================
  .
  .
  ==================================================================
                        Round 4999
  action of player1: 0, palyer2: 1
  belief of player1: [ 179.25 5819.75], player2: [5558.01  440.99]
  payoff of player1: [5819.75  179.25], player2: [ 440.99 5558.01]
  ==================================================================
  strategy distribution of player1: [1. 0.], player2: [0. 1.]
\end{lstlisting}

\newpage

\begin{lstlisting}
                        Round 0
  belief of player1: [963.11  36.89], player2: [165.47 834.53]
  payoff of player1: [ 36.89 963.11], player2: [834.53 165.47]
  ==================================================================
                        Round 1
  action of player1: 1, palyer2: 0
  belief of player1: [964.11  36.89], player2: [165.47 835.53]
  payoff of player1: [ 36.89 964.11], player2: [835.53 165.47]
  ==================================================================
  .
  .
  ==================================================================
                        Round 4999
  action of player1: 1, palyer2: 0
  belief of player1: [5962.11   36.89], player2: [ 165.47 5833.53]
  payoff of player1: [  36.89 5962.11], player2: [5833.53  165.47]
  ==================================================================
  strategy distribution of player1: [0. 1.], player2: [1. 0.]
\end{lstlisting}

\begingroup
\raggedright
If the difference of initial payoff is larger than 1
($|payoff_{r1} - payoff_{r2}| > 1$ and $|payoff_{c1} - payoff_{c2}| > 1$)
The pure-strategy NE $(r_1, c_2)$ and $(r_2, c_1)$ can be found by fictitious paly.
The result will have two situation:

\begin{enumerate}
  \item Once the belief $b_{r1} > b_{r2}, b_{c1} < b_{c2}$,
  player1 will keep choosing $r_1$ rather than $r_2$, 
  and player2 will keep choosing $c_2$ rather than $c_1$, 
  which result in pure-strategy NE $(r_1, c_2)$.
  \item Once the belief $b_{r1} < b_{r2}, b_{c1} > b_{c2}$,
  player1 will keep choosing $r_2$ rather than $r_1$, 
  and player2 will keep choosing $c_1$ rather than $c_2$, 
  which result in pure-strategy NE $(r_2, c_1)$.
\end{enumerate}
\endgroup

\begin{lstlisting}
                        Round 0
  belief of player1: [500.01 499.99], player2: [500.01 499.99]
  payoff of player1: [499.99 500.01], player2: [499.99 500.01]
  ==================================================================
                        Round 1
  action of player1: 1, palyer2: 1
  belief of player1: [500.01 500.99], player2: [500.01 500.99]
  payoff of player1: [500.99 500.01], player2: [500.99 500.01]
  ==================================================================
  .
  .
  ==================================================================
                        Round 4999
  action of player1: 1, palyer2: 1
  belief of player1: [2999.01 2999.99], player2: [2999.01 2999.99]
  payoff of player1: [2999.99 2999.01], player2: [2999.99 2999.01]
  ==================================================================
  strategy distribution of player1: [0.5 0.5], player2: [0.5 0.5]
\end{lstlisting}

\begingroup
\raggedright
If the difference of initial payoff is lower than 1
($|payoff_{r1} - payoff_{r2}| < 1$ and $|payoff_{c1} - payoff_{c2}| < 1$)
and initial strategy distribution of two player is 
$b_{r1} > b_{r2}, b_{c1} > b_{c2}$  or  $b_{r1} < b_{r2}, b_{c1} < b_{c2}$, 
the mixed-strategy NE $P(r_1) = \frac{1}{2}, P(r_2) = \frac{1}{2}, 
P(c_1) = \frac{1}{2}, P(c_2) = \frac{1}{2}$ can be found by fictitious paly.
Since the best response of each iteration will be lead to the cycle 
between $(r_1, c_1)$ and $(r_2, c_2)$.
\endgroup

\newpage

\subsection{Q8: Battle of the Sexes}

\begin{lstlisting}
                        Round 0
  belief of player1: [505.52 494.48], player2: [887.14 112.86]
  payoff of player1: [1516.57  988.96], player2: [1774.27  338.59]
  ==================================================================
                        Round 1
  action of player1: 0, palyer2: 0
  belief of player1: [506.52 494.48], player2: [888.14 112.86]
  payoff of player1: [1519.57  988.96], player2: [1776.27  338.59]
  ==================================================================
  .
  .
  ==================================================================
                        Round 4999
  action of player1: 0, palyer2: 0
  belief of player1: [5504.52  494.48], player2: [5886.14  112.86]
  payoff of player1: [16513.57   988.96], player2: [11772.27   338.59]
  ==================================================================
  strategy distribution of player1: [1. 0.], player2: [1. 0.]
\end{lstlisting}

\begin{lstlisting}
                        Round 0
  belief of player1: [  9.52 990.48], player2: [ 53.65 946.35]
  payoff of player1: [  28.57 1980.95], player2: [ 107.29 2839.06]
  ==================================================================
                        Round 1
  action of player1: 1, palyer2: 1
  belief of player1: [  9.52 991.48], player2: [ 53.65 947.35]
  payoff of player1: [  28.57 1982.95], player2: [ 107.29 2842.06]
  ==================================================================
  .
  .
  ==================================================================
                        Round 4999
  action of player1: 1, palyer2: 1
  belief of player1: [   9.52 5989.48], player2: [  53.65 5945.35]
  payoff of player1: [   28.57 11978.95], player2: [  107.29 17836.06]
  ==================================================================
  strategy distribution of player1: [0. 1.], player2: [0. 1.]
\end{lstlisting}

\begingroup
\raggedright
If the difference of initial payoff is larger than 2
($|payoff_{r1} - payoff_{r2}| > 1$ and $|payoff_{c1} - payoff_{c2}| > 1$)
The pure-strategy NE $(r_1, c_2)$ and $(r_2, c_1)$ can be found by fictitious paly.
The result will have two situation:

\begin{enumerate}
  \item Once the belief $b_{r1} > b_{r2}, b_{c1} < b_{c2}$,
  player1 will keep choosing $r_1$ rather than $r_2$, 
  and player2 will keep choosing $c_2$ rather than $c_1$, 
  which result in pure-strategy NE $(r_1, c_2)$.
  \item Once the belief $b_{r1} < b_{r2}, b_{c1} > b_{c2}$,
  player1 will keep choosing $r_2$ rather than $r_1$, 
  and player2 will keep choosing $c_1$ rather than $c_2$, 
  which result in pure-strategy NE $(r_2, c_1)$.
\end{enumerate}
\endgroup

\newpage

\begin{lstlisting}
                      Round 0
  belief of player1: [399.99 600.01], player2: [600.01 399.99]
  payoff of player1: [1199.97 1200.02], player2: [1200.02 1199.97]
  ==================================================================
                      Round 1
  action of player1: 1, palyer2: 0
  belief of player1: [400.99 600.01], player2: [600.01 400.99]
  payoff of player1: [1202.97 1200.02], player2: [1200.02 1202.97]
  ==================================================================
                      Round 4999
  action of player1: 1, palyer2: 0
  belief of player1: [2399.99 3599.01], player2: [3599.01 2399.99]
  payoff of player1: [7199.97 7198.02], player2: [7198.02 7199.97]
  ==================================================================
  strategy distribution of player1: [0.6 0.4], player2: [0.4 0.6]
\end{lstlisting}

\begingroup
\raggedright
If the difference of initial payoff is lower than 2
($|payoff_{r1} - payoff_{r2}| < 2$ and $|payoff_{c1} - payoff_{c2}| < 2$)
and initial strategy distribution of two player is 
$b_{r1} > b_{r2}, b_{c1} < b_{c2}$  or  $b_{r1} < b_{r2}, b_{c1} > b_{c2}$, 
the mixed-strategy NE $P(r_1) = \frac{3}{5}, P(r_2) = \frac{2}{5}, 
P(c_1) = \frac{2}{5}, P(c_2) = \frac{3}{5}$ can be found by fictitious paly.
Since the best response of each iteration will be lead to the cycle 
between $(r_1, c_2)$ and $(r_2, c_1)$.
\endgroup

\subsection{Q9: Stag Hunt Game}

\begin{lstlisting}
                        Round 0
  belief of player1: [642.99 357.01], player2: [903.66  96.34]
  payoff of player1: [1928.96 1642.99], player2: [2710.97 1903.66]
  ==================================================================
                        Round 1
  action of player1: 0, palyer2: 0
  belief of player1: [643.99 357.01], player2: [904.66  96.34]
  payoff of player1: [1931.96 1644.99], player2: [2713.97 1905.66]
  ==================================================================
  .
  .
  ==================================================================
                        Round 4999
  action of player1: 0, palyer2: 0
  belief of player1: [5641.99  357.01], player2: [5902.66   96.34]
  payoff of player1: [16925.96 11640.99], player2: [17707.97 11901.66]
  ==================================================================
  strategy distribution of player1: [1. 0.], player2: [1. 0.]
\end{lstlisting}

\begin{lstlisting}
                        Round 0
  belief of player1: [234.27 765.73], player2: [144.81 855.19]
  payoff of player1: [ 702.8  1234.27], player2: [ 434.42 1144.81]
  ==================================================================
                        Round 1
  action of player1: 1, palyer2: 1
  belief of player1: [234.27 766.73], player2: [144.81 856.19]
  payoff of player1: [ 702.8  1235.27], player2: [ 434.42 1145.81]
  ==================================================================
  .
  .
  ==================================================================
                        Round 4999
  action of player1: 1, palyer2: 1
  belief of player1: [ 234.27 5764.73], player2: [ 144.81 5854.19]
  payoff of player1: [ 702.8  6233.27], player2: [ 434.42 6143.81]
  ==================================================================
  strategy distribution of player1: [0. 1.], player2: [0. 1.]
\end{lstlisting}

\newpage

\begingroup
\raggedright
If the difference of payoff during the iteration will always larger than 1
($|payoff_{r1} - payoff_{r2}| > 1$ and $|payoff_{c1} - payoff_{c2}| > 1$)
The pure-strategy NE $(r_1, c_2)$ and $(r_2, c_1)$ can be found by fictitious paly.
The result will have two situation:

\begin{enumerate}
  \item Once the belief $b_{r1} > b_{r2}, b_{c1} > b_{c2}$,
  player1 will keep choosing $r_1$ rather than $r_2$, 
  and player2 will keep choosing $c_1$ rather than $c_2$, 
  which result in pure-strategy NE $(r_1, c_1)$.
  \item Once the belief $b_{r1} < b_{r2}, b_{c1} < b_{c2}$,
  player1 will keep choosing $r_2$ rather than $r_1$, 
  and player2 will keep choosing $c_2$ rather than $c_1$, 
  which result in pure-strategy NE $(r_2, c_2)$.
\end{enumerate}
\endgroup

\begin{lstlisting}
                        Round 0
  belief of player1: [500.01 499.99], player2: [499.99 500.01]
  payoff of player1: [1500.03 1500.01], player2: [1499.97 1499.99]
  ==================================================================
                        Round 1
  action of player1: 0, palyer2: 1
  belief of player1: [500.01 500.99], player2: [500.99 500.01]
  payoff of player1: [1500.03 1501.01], player2: [1502.97 1501.99]
  ==================================================================
  .
  .
  ==================================================================
                        Round 4999
  action of player1: 0, palyer2: 1
  belief of player1: [2999.01 2999.99], player2: [2999.99 2999.01]
  payoff of player1: [8997.03 8998.01], player2: [8999.97 8998.99]
  ==================================================================
  strategy distribution of player1: [0.5 0.5], player2: [0.5 0.5]
\end{lstlisting}

\begingroup
\raggedright
If the difference of initial payoff is lower than 1
($|payoff_{r1} - payoff_{r2}| < 1$ and $|payoff_{c1} - payoff_{c2}| < 1$)
and initial strategy distribution of two player is 
$b_{r1} > b_{r2}, b_{c1} < b_{c2}$  or  $b_{r1} < b_{r2}, b_{c1} > b_{c2}$, 
the mixed-strategy NE $P(r_1) = \frac{1}{2}, P(r_2) = \frac{1}{2}, 
P(c_1) = \frac{1}{2}, P(c_2) = \frac{1}{2}$ can be found by fictitious paly.
Since the best response of each iteration will be lead to the cycle 
between $(r_1, c_2)$ and $(r_2, c_1)$.
\endgroup

\subsection{Q10: Observation and Conclusion}

to be continue...

% \begin{figure*}[tbh]
%     \includegraphics[width=\textwidth]{"./student_forum/SAGAN_flow.pdf"}
%     \caption{The architecture of SAGAON.}
%     \label{fig:SAGAON}
% \end{figure*}

% \begin{figure}[tbh]
%     \centering
%     \begin{subfigure}{.5\columnwidth}
%       \centering
%       \includegraphics[width=\linewidth]{"./student_forum/station_level_RMSE.pdf"}
%       \caption{Station-level RMSE}
%     \end{subfigure}%
%     \hfill
%     \begin{subfigure}{.5\columnwidth}
%       \centering
%       \includegraphics[width=\linewidth]{"./student_forum/station_level_MAPE.pdf"}
%       \caption{Station-level MAPE}
%     \end{subfigure}%
%     \caption{SAGAON compares with different baseline models.}
%     \label{fig2}
% \end{figure}

\end{document}